% Created 2024-04-01 Mon 13:41
% Intended LaTeX compiler: pdflatex
\documentclass[11pt]{article}
\usepackage[utf8]{inputenc}
\usepackage[T1]{fontenc}
\usepackage{graphicx}
\usepackage{longtable}
\usepackage{wrapfig}
\usepackage{rotating}
\usepackage[normalem]{ulem}
\usepackage{amsmath}
\usepackage{amssymb}
\usepackage{capt-of}
\usepackage{hyperref}
\usepackage{placeins}
\usepackage{gensymb}
\author{Christian Johnson /and Aidan Andersen}
\date{\today}
\title{Project 2}
\hypersetup{
 pdfauthor={Christian Johnson /and Aidan Andersen},
 pdftitle={Project 2},
 pdfkeywords={},
 pdfsubject={},
 pdfcreator={Emacs 29.3 (Org mode 9.6.15)}, 
 pdflang={English}}
\begin{document}

\maketitle
\tableofcontents

\newpage



\section{Python Code}
\label{sec:org25cd054}
\begin{verbatim}
from math import factorial

def Pascal(numRows):
    triangle=""
    for i in range(numRows+1):
        for j in range(numRows-i+1):
            triangle+="  "
        # loop to get elements of row i
        for j in range(i+1):
            # nCr = n!/((n-r)!*r!)
            triangle+=( str(factorial(i)//(factorial(j)*factorial(i-j))) + "   " )

        triangle+="\n"

    return triangle
\end{verbatim}

\section{Print Pascal's Triangle}
\label{sec:orgb60caa6}

\subsection{\((s-t)^{10}\)}
\label{sec:orgc6341a1}

\begin{verbatim}
Pascal(10)
\end{verbatim}

\begin{verbatim}
                    1   
                  1   1   
                1   2   1   
              1   3   3   1   
            1   4   6   4   1   
          1   5   10   10   5   1   
        1   6   15   20   15   6   1   
      1   7   21   35   35   21   7   1   
    1   8   28   56   70   56   28   8   1   
  1   9   36   84   126   126   84   36   9   1   
1   10   45   120   210   252   210   120   45   10   1   
\end{verbatim}

\textbf{Binomial Expansion}:
\(s^{10}+10s^{9}t+45s^{8}t^2+120s^7t^3+210s^6t^4+252s^5t^5+210s^4t^6+120s^3t^7+45s^2t^8+10st^9+t^{10}\)

\subsection{\((2x+y)^5\)}
\label{sec:org30ec46f}
\begin{verbatim}
Pascal(5)
\end{verbatim}

\begin{verbatim}
          1   
        1   1   
      1   2   1   
    1   3   3   1   
  1   4   6   4   1   
1   5   10   10   5   1   
\end{verbatim}


\textbf{Binomial Expansion}:
\(2x^{5}+5(2x)^{4}y+10(2x)^{3}y^{2}+10(2x)^{2}y^{3}+5(2x)y^{4}+y^{5}\)

\section{List elements of a power set}
\label{sec:org9c7ac69}

\subsection{Python}
\label{sec:org0c932de}
\begin{verbatim}
from itertools import chain, combinations

def powerset(given):
    s=list(given)
    result=chain.from_iterable(combinations(s,r) for r in range(len(s)+1))
    return result

# example
my_set={1,2,3,4}
str(list(powerset(my_set)))
\end{verbatim}

\begin{center}
\begin{tabular}{rrll}
1 &  &  & \\[0pt]
2 &  &  & \\[0pt]
3 &  &  & \\[0pt]
4 &  &  & \\[0pt]
1 & 2 &  & \\[0pt]
1 & 3 &  & \\[0pt]
1 & 4 &  & \\[0pt]
2 & 3 &  & \\[0pt]
2 & 4 &  & \\[0pt]
3 & 4 &  & \\[0pt]
1 & 2 & 3 & \\[0pt]
1 & 2 & 4 & \\[0pt]
1 & 3 & 4 & \\[0pt]
2 & 3 & 4 & \\[0pt]
1 & 2 & 3 & 4\\[0pt]
\end{tabular}
\end{center}
\end{document}
